Some comments:



In the introduction: an algebraic variety in general is a morphism of finite type to Spec K where K is a field. There is no mention of a field in the second paragraph, where varieties are first discussed. : Fixed.



After definition 1, you use
Hom_n...
but I do not see that you tell the reader what the purpose of n is? : Fixed.



Later, you talk about the projective line P^1. What is this object? Is it a variety? Over what field, ring? I would recommend that you be more precise here, and write P^1_S whenever you mean the projective line over S, and maybe P^1_K when S=Spec K. : Fixed.



Theorem 2: The first sentence seems to be unfinished, and the if is not followed by a then. : Fixed.



Your write Stck_K in two different ways: with K-bold, and with K-normal. : Fixed.



Definition 3: I find the notation unpleasant:
|X| neans a set, while |Stab(F_q)| means the number of elements in a set. : I think you fixed it?



Then in Corollary 4, you have
... \leq |Hom| \leq
and there I am really confused as to what
|Hom| means. : Not sure how to fix.



Page 2:  Classification of surfaces. What field are we working on? Algebraically closed of any characteristic? Perfect? : Not sure how to fix.


with higher genus fibers... higher than what? : We mention \ge 2.



...Renders the classification intractable: I am attaching two papers for the classification when g=3. : Fair enough, I added the references but it is still true that brute force classification cannot continue to genus \ge 4.



In definition 5, you seem to assume that the curve is projective when you count the ramification points, but you do not state that hypothesis. : Fixed.



Corollary 8 is hard to parse. Make several sentences...
These are isomorphisms in what categories? Later it is an equality in what set? What is the quantifier on the symbol n? : I think you fixed the latter parts. We can break this into several sentences.



Problem 9: first sentence is missing something.
Why call this a problem and then say that it is a conjecture, and not state what the conjecture actually is? : The conjecture was that it is finite. Shafarevich himself showed that it is finite for genus 1 elliptic case. I changed Shafarevich's conjecture to Shafarevich's problem.



Through the global... the sentence is not making sense as written. : I explain how is it that they share many properties by the various connections and similarities between the theory of algebraic functions of one variable and the theory of algebraic numbers.



Theorem 10 is hard to parse. 
What is Z_{g,..}(B)? : One should read the Section 5, more specifically Page 23.
What is B? : Again, one should read the Section 5, more specifically Page 23.
What is a sharp asymptotic? : I'm not sure how to reply, we are counting with respect to height which in this section is the cardinality of the residue field. Obviously we acquire an upper bound which is `asymptotic' and it is `sharp' since we actually have an equality at precisely certain height (which is the height of the Discriminant.)



page 6: of the the rational : Fixed.